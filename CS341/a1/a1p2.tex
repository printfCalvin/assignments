\documentclass[11pt]{article}
\usepackage{fullpage,amsmath, graphicx}
\usepackage[linesnumbered,ruled,vlined]{algorithm2e}
\renewcommand{\baselinestretch}{1.02}
\newcommand{\Q}[1]{\medskip\item {[{\em #1 marks\/}]}\ }
\usepackage{xcolor}
\newif\ifsol
\soltrue
\newcommand{\solution}[1]{{\ifsol \color{red} {#1} \fi}}


\begin{document}
%\thispagestyle{empty}

\hfill CS 341, Spring 2020\par
\hfill Semih Salihoglu

\bigskip
\begin{center}\large\bf Assignment 1 Part 2 (due Sunday, May 31, midnight EST)
\end{center}

\noindent{\bf Instructions:}
\begin{itemize}
\item Hand in your assignment using Crowdmark. Detailed instructions are on the course website.
\item Give complete legible solutions to all questions.
\item Your answers will be marked for clarity as well as correctness.
\item For any algorithm you present, you should justify its correctness
(if it is not obvious) and analyze the complexity.
\end{itemize}

\begin{enumerate}


\Q{15}
A group of hackers from an enemy organization has attempted to install a virus to $n$ of your company's computers. Your software engineers have designed a test, called \texttt{TEST-EACH-OTHER}, that takes two computers $c_A$ and $c_B$, where each input computer tests the other and outputs whether the other one is infected with the virus ($+$)  or not infected with the virus ($-$).     
If a computer is actually $-$ than it will always output a correct result. Unfortunately,
if it is $+$, its reply is unrelated to the real state of the other computer and hence cannot be trusted. In other words, a computer $c_A$ that is infected with the virus can be ``dishonest'' and output the correct or the incorrect state of $c_B$. 

The following table summarizes the four possible outcomes of running \texttt{TEST-EACH-OTHER} on two computers $c_A$ and $c_B$, and what we can conclude from it. Please review the table to ensure that these are indeed the possible outcomes.


\begin{tabular}{ p{3cm} | p{3cm} | p{5cm}  }
{\bf $c_A$'s output} & {\bf $c_B$'s output} &{\bf Conclusion}  \\
\hline
\cline{1-3}\cline{1-3}\cline{1-3}\cline{1-3}\cline{1-3}\cline{1-3}
$c_B$ is $-$ & $c_A$ is $-$ &  either both $-$ or both $+$ \\
\cline{1-3}\cline{1-3}\cline{1-3}\cline{1-3}\cline{1-3}\cline{1-3}
$c_B$ is $-$ & $c_A$ is $+$ & at least one is $+$ \\
\cline{1-3}\cline{1-3}\cline{1-3}\cline{1-3}\cline{1-3}\cline{1-3}
$c_B$ is $+$ & $c_A$ is $-$ & at least one is $+$ \\
\cline{1-3}\cline{1-3}\cline{1-3}\cline{1-3}\cline{1-3}\cline{1-3}
$c_B$ is $+$ & $c_A$ is $+$  & at least one is $+$ \\
\cline{1-3}
\end{tabular}

\leavevmode 
\\\\
 Luckily your security experts have told you that more than $n/2$ computers were not infected (so they are $-$). Your goal is to identify all the $+$ and $-$ computers. Below, running one instance of \texttt{TEST-EACH-OTHER} constitutes one test.

 \begin{enumerate}
  \Q{12} Describe an algorithm to find a single $-$ phone by performing $O(n)$ tests. [Hint: Think of how you can use $O(n)$ tests to reduce the problem size by a constant factor.]
  
  \solution{
        We first partition all computers in to two equally sized groups or two groups with size difference of 1
        if $n$ is odd. Let $A$ and $B$ be the two groups. Then run \texttt{TEST-EACH-OTHER} using the $i$-th 
        computer in $A$ $a_i$ to test the $i$-th computer in $B$ $b_i$. And using $b_i$ to test $a_i$. If the 
        outputs are both $-$, put them into a new set $D$. And we pick the first element of each pair and 
        do the same process above, until there is less than $2$ elements left. \\
        We assume \texttt{TEST-EACH-OTHER}($c_A, c_B$) is equivalent to use $c_A$ to test $c_B$.\\
        \begin{algorithm} [H]
                \caption{FindUninfectedCmptr($c[1, \dots, n]$)}
                \If {$size(c) \leq 2$} {
                        \Return $c[1]$
                }
                $A = c[1, \dots, \lfloor n / 2 \rfloor]$\\
                $B = c[\lfloor n / 2 \rfloor + 1, \dots, n]$\\
                $D = \emptyset$\\
                \For {$c_a \in A$ and $c_b \in B$} {
                        \If {\texttt{TEST-EACH-OTHER}$(c_a, c_b) == - $ and 
                        \texttt{TEST-EACH-OTHER}$(c_b, c_a) == - $} {
                                $D = D \cap \{c_a\}$\\
                        }
                }
                \If {$\lvert D \rvert$ is even and $\lvert c \rvert$ is odd} {
                        $D = D \cap \{ c[n] \}$
                }
                \Return FindUninfectedCmptr($D$)
        \end{algorithm}
        Every time with results are not $--$, we remove one uninfected computer and one infected or two 
        infected computers. So at least $\frac 1 2$ of the computers in the new set are uninfected.
        Since there are more that $n / 2$ computers are not infected. There is at least one piar where both 
        computers are $-$. $D$ is not empty if $n$ is even. If $n$ is odd, $D$ is even and there are two 
        more $-$ computers than $+$ computers, the number of $-$ computers is more than the number of $+$ 
        computers after adding any computer with either $-$ or $+$. If $n$ is odd, $D$ is even and the number 
        of $+$ computers and the number of $-$ computers is equal, then the single computer is $-$, since 
        ther are more than $n/2$ computers were not infected. Every time we reduce the set to at 
        least $\frac 1 2$ of its original size, we will eventually have a set with size at most $4$. Hence, 
        the program always terminates and return the correct output.

        \textbf{Analysis:} Every time we left with at most $\lfloor \frac n 2 \rfloor + 1$ of the 
        computers and call \texttt{TEST-EACH-OTHER} $2n$ times. Let $T(n)$ be the number of tests of $n$ computers.
        \begin{align*}
                T(n) & \leq 2 n + (2 \times \frac n 2 + 1) + (2 \times \frac n 4 + 1) + \dots + 2 + 1\\
                & = 2(n + \frac n 2 + \frac n 4 + \dots + \frac {n} {\frac n 2}) + \log_2 {\frac n 2} + 1\\
                & = 2 n \frac {1 - (\frac 1 2)^{\log_2 \frac n 2 + 2}} {1- \frac 1 2} + \log_2 n - \log_2 2 + 1\\
                & = 4 n (1 - (\frac 1 2)^{\log_2 n + 1}) + \log_2 n\\
                & = 4 n - 2 + \log_2 n\\
                & \in O(n)
        \end{align*}
  }
  
  \Q{3} Using part (a), show how to identify the condition of each computer by performing $O(n)$ tests.

\solution{
        We first use the algorithm from (a) to find a single computer $c$ that is not infected and call  
        \texttt{TEST-EACH-OTHER} using $c$ to determine whether other computers are infected. Since finding 
        $c$ costs $O(n)$ tests and using $c$ to determine the other computers' status uses $n - 1$ tests, so 
        the overall runtime is $O(n)$.
}
  \end{enumerate} 


\newpage
\Q{12}  Consider the recurrence:
$$T(n) = 2T(\lfloor n/9 \rfloor) + \sqrt{n} \text{ \quad\quad if } n \ge 9$$  
$$T(n) = 5 \text{ \quad\quad\quad\quad\quad\quad\quad\quad\quad if } n < 9$$
Prove $T(n) = O(\sqrt{n})$ by induction (i.e., guess-and-check or substitution method).  Show what your $c$ and $n_0$ are in your big-oh bound.  Note that depending on the choice of your $n_0$, you might have to cover multiple base cases in your inductive proof.

\solution{
        Assume T$(n) \leq 5 \sqrt{n}$\\
        \textbf{Base Case:} Assume $k < 9$, $T(k) = 5 \leq 5 \sqrt{k}$\\
        \textbf{IH:} Assume $T(k) \leq 5 \sqrt{n}$. for all $k \leq n - 1$\\
        \textbf{IS:}
        \begin{align*}
                T(n) & = 2T(\lfloor n/9 \rfloor) + \sqrt{n}\\
                & = 2(5 \sqrt{\lfloor n/9 \rfloor}) + \sqrt{n}\\
                & \leq \frac {10} {\sqrt{9}} \sqrt{n} + \sqrt{n}\\
                & \leq 5 \sqrt{n}
        \end{align*}
        Let $c = 6$, $n_0 = 1$.\\
        Clearly, $5\sqrt{n} \leq 6 \sqrt{n}$ for all $n \geq n_0$. Hence, $T(n) \leq 5 \sqrt{n} \in O(\sqrt{n})$ 
}
 

\newpage
\Q{16} 
Give tight asymptotic ($\Theta$) bounds for the solution 
to the following recurrences by using the recursion-tree method
or the induction method (your choice).
You may assume that $n$ is a power of $10$ in (a), or a power of
$3$ in (b).  Show your work.  
\begin{enumerate}
\Q{8}
\[ T(n)=\left\{\begin{array}{ll}
        2\,T(n/10)+\sqrt{n}  & \mbox{if $n>1$}\\
        7 & \mbox{if $n\le 1$}
        \end{array}\right.
\]

\solution{
        \begin{align*}
                T(n) & = \sqrt{n} + 2 \sqrt{\frac {n} {10}} + 4 \sqrt{\frac {n} {100}} + 
                \dots + 7 \times 2^{\log_{10} n}\\
                & = \sqrt{n} (\frac {2} {\sqrt{10}} + (\frac {2} {\sqrt{10}})^2 + \dots + 
                (\frac {2} {\sqrt{10}})^{\log_{10} n - 1}) + 7 \times 2^{\log_{10} n}\\
                & = \sqrt{n} \times \Theta(1) + 7 n^{\log_{10} 2}\\
                & \in \Theta(n^{\frac 1 2})
        \end{align*}
}

\Q{8}
\[ T(n)=\left\{\begin{array}{ll}
         10\,T(n/3)+n^2 & \mbox{if $n>1$}\\
         1 & \mbox{if $n\le 1$}
         \end{array}\right.
\]

\solution{
        \begin{align*}
                T(n) & = n^2 + 10 \times (\frac n 3)^2 + 100 \times (\frac n 9)^2 + \dots 
                + 1 \times 10^{\log_3 n}\\
                & = n^2 (1 + \frac {10} {9} + (\frac {10} {9})^2 + \dots + (\frac {10} {9})^{\log_3 n - 1}) +
                10^{\log_3 n}\\
                & = n^2 \times 9 ((\frac {10} {9})^{\log_3 n} - 1) + n^{\log_3 10}\\
                & = 9 n^{\log_3 \frac {10} {9} + 2} - 9 n^2 + n^{\log_3 10}\\
                & = 10 n^{\log_3 10} - 9 n^2\\
                & \in \Theta(n^{\log_3 10})
        \end{align*}
}

\end{enumerate}

\newpage
\Q{6} 
\begin{enumerate}
\item Solve part (a) of the previous question by the
master method.

\solution{
        $a = 2, b = 10, c = 1 / 2$. $2 < 10^{1 / 2}$ by master method $T(n) \in \Theta(\sqrt{n})$
}

\item Solve part (b) of the previous question by the
master method.

\solution{
        $a = 10, b = 3, c = 2$. $10 > 3^{2}$ by master method $T(n) \in \Theta(n^{\log_3 10})$
}

\end{enumerate}

\end{enumerate}





\end{document}

