\documentclass[11pt]{article}
\usepackage{fullpage,amsmath,mathtools}
\usepackage[linesnumbered,ruled,vlined]{algorithm2e}
\renewcommand{\baselinestretch}{1.02}
\newcommand{\Q}[1]{\medskip\item {[{\em #1 marks\/}]}\ }
\usepackage{xcolor}
\newif\ifsol
\soltrue
\newcommand{\solution}[1]{{\ifsol \color{red} {#1} \fi}}


\begin{document}
%\thispagestyle{empty}

\hfill CS 341, Spring 2020\par
\hfill Semih Salihoglu

\bigskip
\begin{center}\large\bf Assignment 1 Part 1 (due Sunday, May 24, midnight EST)
\end{center}

\noindent{\bf Instructions:}
\begin{itemize}
\item Hand in your assignment using Crowdmark. Detailed instructions are on the course website.
\item Give complete legible solutions to all questions.
\item Your answers will be marked for clarity as well as correctness.
\item For any algorithm you present, you should justify its correctness
(if it is not obvious) and analyze the complexity.
\end{itemize}

\begin{enumerate}


\Q{8} Give a proof from first principles (not using limits) the following statements: 
\begin{enumerate}
\Q{4} 
$$n^{2.7} - 100n^{2.4} + 1000 \in \omega(n^{2.5})$$


\solution{
    We want to prove for all constant $c > 0$, there exists a $n_0$ such that 
    $\lvert n^{2.7} - 100n^{2.4} + 1000 \rvert > \lvert n^{2.5} \rvert$ for all $n > n_0$\\
    If $c > 1$\\
    Let $n_0 = 10^{10} c^{10}$ and $n \geq n_0$
    \begin{align*}
        \lvert n^{2.7} - 100n^{2.4} + 1000 \rvert &= n^{2.5} (n^{0.2} - \frac {100} {n^{0.1}} + \frac {1000} {n^{2.5}})\\
        &\geq n^{2.5} (10^2 c^2 - \frac {100} {n^{0.1}} + \frac {1000} {n^{2.5}})\\
        &\geq n^{2.5} (10^2 c^2 - \frac {100} {10c} + \frac {1000} {n^{2.5}})\\
        &= n^{2.5} (10^2 c^2 - \frac {10} {c} + \frac {1000} {n^{2.5}})\\
        &> n^{2.5} (10^2 c^2 - \frac {10} {c} + 1)
        \text{ since $\frac {1000} {n^{2.5}} \leq \frac {1000} {10^{25} c^{25}} = \frac 1 {10^{22} c^{25}} < 1$}\\
        &> c n^{2.5}\\
        &> 0
    \end{align*}
    If $0 < c \leq 1$
    Let $n_0 = 10^{10}$ and $n \geq n_0$
    \begin{align*}
        n^{2.7} - 100n^{2.4} + 1000 &= n^{2.5} (n^{0.2} - \frac {100} {n^{0.1}} + \frac {1000} {n^{2.5}})\\
        &\geq n^{2.5} (10^2 - \frac {100} {10} + \frac {1000} {n^{2.5}})\\
        &< n^{2.5} (100 - 10 + 1) \text{ since $\frac {1000} {n^{2.5}} \leq \frac 1 {10^{22}} < 1$}\\
        &= 91 n^{2.5}\\
        &> c\\
        &> 0
    \end{align*}
    Hence, $n^{2.7} - 100n^{2.4} + 1000 \in \omega(n^{2.5})$
}


\Q{4} Let $f(n)$ and $g(n)$ be positive-valued functions. Then: 
$$\max\{f(n), g(n)\}=\Theta(f(n) + g(n))$$

\solution{
    We want to prove that there exists $c_1 > 0, c_2 > 0$ such that 
    $$c_1 \lvert (f(n) + g(n)) \rvert \leq \lvert \max\{f(n), g(n)\} \rvert 
    \leq \lvert c_2 (f(n) + g(n)) \rvert$$ for all $n \leq n_0$
    Let $c_1 = \frac 1 2, c_2 = 1$
    \begin{align*}
        2\max\{f(n), g(n)\} &\leq f(n) + g(n)\\
        \text{Hence, }\max\{f(n), g(n)\} &\leq \frac 1 2 (f(n) + g(n)) \text{ for all n}
    \end{align*}
    If $\max\{f(n), g(n)\} = f(n)$, \\$\max\{f(n), g(n)\} = f(n) < f(n) + g(n)$ since $g(n) > 0$ for all $n$\\
    If $\max\{f(n), g(n)\} = g(n)$, \\$\max\{f(n), g(n)\} = g(n) < f(n) + g(n)$ since $f(n) > 0$ for all $n$\\
    Therefore $\max\{f(n), g(n)\}=\Theta(f(n) + g(n))$
}

\end{enumerate}

\Q{12} 
For each pair of functions $f(n)$ and $g(n)$, 
fill in the correct asymptotic notation among $\Theta$,
$o$, and $\omega$ in the statement $f(n)\in$ \verb*| | $(g(n))$.
Formal proofs are not necessary, but provide brief justifications
for all of your answers. (The default base in logarithms is 2.)
\begin{enumerate}
\item $f(n) = (8n)^{250} + (3n + 1000)^{500}$ \ \ vs.\ \ $g(n)=n^{500} + (n+1000)^{400}$

\solution{
    $f(n)\in \Theta(g(n))$\\
    $\lim_{n \to \infty} \frac {f(n)} {g(n)} = 3^{500}$
}

\item $f(n)= n^{1.5}2^n $\ \ vs.\ \ $g(n)= (n)^{100}1.99^n$.

\solution{
    $f(n)\in \omega (g(n))$\\
    $$\lim_{n \to \infty} \frac {n^{1.5}2^n} {n^{100}1.99^n} = \frac {(\frac 2 {1.99})^n} {n^{98.5}} 
    \overset {L'H} {=} \infty$$
}


\item $f(n)=(256)^{n/4}$\ \ vs.\ \ $g(n)=(125)^{n/3}$

\solution{
    $f(n)\in o (g(n))$\\
    $$\lim_{n \to \infty} \frac {256^{n/4}} {125^{n/3}} = (\frac 4 5)^n = 0$$
}

\item $f(n) = 2^{\log(n) \cdot \log(n)} $\ \ vs.\ \ $g(n)=n^{2012}$

\solution{
    $f(n)\in \omega (g(n))$\\
    $f(n) = 2^{\log(n) \cdot \log(n)} = n^{\log(n)}$\\
    $\lim_{n \to \infty} \frac {n^{\log(n)}} {n^{2012}} >\lim_{n \to \infty} \frac {n^2013} {n^{2012}}
    =\lim_{n \to \infty} n=\infty$
}

\end{enumerate}

\newpage 
\Q{10}
Analyze the following pseudocodes and give a tight $\Theta$
bound on the running time as a function of $n$.  Carefully show your work.

\begin{enumerate}
\Q{5} 
\begin{tabbing}
9.M\= for\= for\=\ for\=\+\kill
1.\' for $i=1$ to $n$ do \\
2.\'\>A$[i]$ = true\\  
3.\' for $i=1$ to $n$ do \\
4.\'\> $j=i$\\
5.\'\> while $j \le n$ do\\
6.\'\>\>A$[j]$ = false\\
7.\'\>\> $j=j+i$\\
\end{tabbing}

\solution{
    \begin{align*}
        T(n) & = n + \sum_{i = 1}^{n} \lfloor \frac {2 n} {i} \rfloor\\
        & \leq n + \sum_{i = 1}^{n} \frac {2 n} {i}\\
        & = n + 2n \Theta (\log n)\\
        & \in O(n \log n)
    \end{align*}
Also, 
    \begin{align*}
        T(n) & = n + \sum_{i = 1}^{n} \lfloor \frac {2 n} {i} \rfloor\\
        & \geq n + \sum_{i = 1}^{n} \frac {1} {i}\\
        & = n + n \Theta (\log n)\\
        & \in \Omega(n \log n)
    \end{align*}
Hence, $T(n) \in \Theta(n \log n)$
}


\Q{5}
The following is a sorting algorithm that sorts an array $A$ of $n$ integers, where each integer $e_i \in A$ is $0 \le e_i \le m-1$.  Go through the code and verify that this algorithm indeed sorts A correctly.
\begin{tabbing}
9.M\= for\= for\=\ for\=\+\kill
1.\' for $i=0$ to $m-1$ do\\
2.\'\> counts$[i]=0$\\
3.\' for $i=0$ to $n-1$ do\\
4.\'\> counts$[A[i]]++$\\
5.\' $k=0$\\
6.\' for $i=0$ to $m-1$ do\\
7.\'\> for $j=0$ to counts$[i]-1$ do\\
8.\'\>\> $A[k]=i$, $k=k+1$
\end{tabbing}

\solution{
    line 2 execuates $m$ times, line 4 execuates $n$ times, line 8 execuates exactly $n$ time since $A$ has 
    size $n$.\\
    Putiing all together,
    \begin{align*}
        T(n) & = m + n + 1 + n\\
        & \in \Theta (\max(m, n))
    \end{align*}
}

\end{enumerate}

\Q{12}
Given a string $s = a_1 a_2 ... a_n$ of length $n$, where $a_1 a_2 ... a_n \in \{0, 1\}$, decide whether $s$ is the $k$th {\em power} of a sub-string $t$, i.e., $s=t^k$, for some $k > 1$ and string $t$. Here, $t^k$ denotes the string $t$ repeated $k$ times. For example, 01000100, 10101010, and 000000, are all perfect powers (e.g. 01000100 = 0100$^2$) but  01000110 is not. 

Give an algorithm that solves this problem in $O(n^{3/2})$ time. Describe your algorithm, provide the pseudocode, and analyze the run-time of your algorithm.

Hint: Observe that if $s=t^k$, and $t$ has length $\ell$, then $n = \ell k$. This implies that $\ell$ and $k$ cannot both be greater than $\sqrt{n}$.


\end{enumerate}

\solution{
    IsPerfectPwoer(s) first loops through $2$ to $\sqrt{n}$ to find all possible values of $l$ that divides $n$.
    For each valid value $l$, it checks whether each sub-string with length $l$ is actually equal. Then, it 
    switches the value of $l$ and $k$, i.e. let $l = n / l$ and do the exactly same process to check if all 
    substring are equal.\\
    \begin{algorithm} [H]
        \caption{IsPerfectPwoer(s)}
        \For {$i = 2$ to $\sqrt{n}$} {
            $l = i$\\
            \If {$l$ divides $n$} {
                \For {$j = 1$ to $n / l - 1$} {
                \If {\upshape s[$j \times l \dots (j + 1) \times l $] != s[$0 \dots l - 1$]} {
                        \textbf{break}\\
                    }
                \If {$j == n / l - 1$} {
                    \Return True\\
                }
                }
                $l = n / l$\\
                \For {$j = 1$ to $n / l - 1$} {
                    \If {\upshape s[$j \times l \dots (j + 1) \times l $] != s[$0 \dots l - 1$]} {
                        \textbf{break}\\
                    }
                    \If {$j == n / l - 1$} {
                        \Return True\\
                    }
                }
            }
        }
        \Return False\\
    \end{algorithm}
    \textbf{Run-time analysis:} The outer loop(line 1-14) iterates at most $\sqrt{n}$ times. Both inner 
    loops(line 4-8, 10-14) iterates at most $n / l - 1$ times.\\
    Hence, 
    \begin{align*}
        T(n) & \leq \sqrt{n} \times 2 (\frac n l - 1)\\
        & \leq 2 n^{\frac 3 2} - n^{\frac 1 2} \text{ , since } l \geq 1\\
        & \in O(n^{\frac 3 2})
    \end{align*}
}

\end{document}
