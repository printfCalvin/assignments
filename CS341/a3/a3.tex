\documentclass[11pt]{article}
\usepackage{fullpage,amsmath, graphicx}
\usepackage[linesnumbered,ruled,vlined]{algorithm2e}
\usepackage{amsthm}
\renewcommand{\baselinestretch}{1.02}
\newcommand{\Q}[1]{\medskip\item {[{\em #1 marks\/}]}\ }
\usepackage{xcolor}
\newif\ifsol
\soltrue
\newcommand{\solution}[1]{{\ifsol \color{red} {#1} \fi}}

\newtheorem{claim}{Claim}

\newcommand{\down}[1]{\left\lfloor #1\right\rfloor}
\newcommand{\up}[1]{\left\lceil #1\right\rceil}

\begin{document}

\hfill CS 341, Spring 2020\par
\hfill Semih Salihoglu

\bigskip
\begin{center}\large\bf Assignment 3 (due Thursday, July 2nd, midnight EST)
\end{center}

\noindent{\bf Instructions:}
\begin{itemize}
\item Hand in your assignment using Crowdmark. Detailed instructions are on the course website.
\item Give complete legible solutions to all questions.
\item Your answers will be marked for clarity as well as correctness.
\item For any algorithm you present, you should justify its correctness
(if it is not obvious) and analyze the complexity.
\end{itemize}

\begin{enumerate}

\Q{25} {\em Programming Question:} This is a programming question. Details on how you will submit your code will be posted on Piazza.

Gerrymandering is the practice of carving up electoral
districts in very careful ways so as to lead to outcomes that favor a particular political party.
This (terrible) practice is effectively the solution to the following computational problem.
Suppose we have a set of $n$ precincts (with $n$ even), each containing $m$ registered voters and two political parties: party $A$ and party $B$. Suppose further that for each precinct, we have information on how many voters are registered to each of two political parties. You can assume each voter is registered to one of the parties.  
We say that the set of precincts is susceptible to gerrymandering if it is possible
to divide the precincts into 2 districts, each consisting of $n/2$ of the precincts in such a way that the same party holds a majority in both districts.

\paragraph{Example.} Suppose we have $n = 4$ precincts, and the following information on registered voters.
\begin{quote}
\begin{tabular}{l|cccc}
Precinct &1& 2& 3& 4\\\hline
Number registered for party A &55 &43 &60 &47\\
Number registered for party B &45 &57 &40 &53
\end{tabular}
\end{quote}
This set of precincts is susceptible since, if we grouped precincts 1 and 4 into one district,
and precincts 2 and 3 into the other, then party A would have a majority in both districts.
(Presumably, the ``we'' who are doing the grouping here are members of party A.) This
example is a quick illustration of the basic unfairness in gerrymandering: Although party A
holds only a slim majority in the overall population (205 to 195), it ends up with a majority
in not one but both districts.

In this problem, you are a software engineer working for party $A$ (of course, do not work for such parties in real life). Your program will be given an $n$-size array $A[n]$ and a number $m$ as input, where $A[k]$ stores the number of registered voters for party $A$ in precinct $k$. So the number of registered voters for party $B$ in precinct $k$ is $m-A[k]$. Your goal is to write an $O(mn^3)$-time dynamic programming algorithm (so quite slow) that solves the gerrymandering problem. That is, let $P=\{1,...,n\}$ be the set of all precincts. Your algorithm should return true if there exists a subset $S\subset P$ of size $n/2$ such that in both $S$ and $P \setminus S$, party A holds a majority (i.e., if $\sum_{k \in S} a_k > nm/4$ and $\sum_{k \in P \setminus S} a_k > nm/4$).

We will give you the recurrence you need to get you stared. Aside from programming the algorithm, you need to think about what your base cases should be and how to return the final answer. Let $a_k$ denote the number of voters registered for party A in precinct $k$.  (Then $m-a_k$ is the number of voters registered for party B in precinct $k$.)
Define the following subproblems ($i=0,\ldots,n$, $j=0,\ldots,n$, $\ell=0,\ldots,mn$):
\[ C[i,j,\ell] = \left\{\begin{array}{ll}
1 & \mbox{if there exists a subset $S\subseteq\{1,\dots,i\}$ such that $|S|=j$ and $\sum_{k\in S}a_k=\ell$}\\
0 & \mbox{else}
\end{array}\right.
\]

Forgetting about the base cases, $C[i, j, \ell]$ can be expressed through the following recurrence:
$$C[i,j,\ell] = C[i-1,j,\ell] \vee C[i-1,j-1,\ell-a_i].$$
(For $\ell<a_i$, just take $C[i,j,\ell]=C[i-1,j,\ell]$.)

This is because, there is a subset $S$ of size $j$ from precincts $1,...,i$ such that party $A$'s total votes in $S$ sum to $\ell$ either: (i) when precinct $i \not \in S$ and there is a subset $S'$ from precincts $1,...,i-1$ that sum to exactly $\ell$; or (ii) when precinct $i \in S$ and and there is a subset $S'$ from precincts $1,...,i-1$ that sum to exactly $\ell - a_i$; 

Using the above recurrence, write a complete and correct dynamic programming algorithm that returns true (if party A can obtain a majority in both districts through gerrymandering) or false (otherwise). In addition, if your algorithm returns true, it should also return such a subset $S$.

You can either use Python2 or C++. Your program should read from standard input and print to standard output. The input consists of two lines. The first contains the two integers $n,m$ separated by a space. The second contains the $n$ integers entries of matrix $A[n]$ separated by a space. Your program should print either {\tt true} or {\tt false}. In the positive case, it should also print in a new line $n/2$ space-separated integers from the set $\{1,\ldots,n\}$ that represent the subset $S$.\\
Input sample:\\
{\tt 4 100}\\
{\tt 55 43 60 47}\\
Output sample:\\
{\tt true}\\
{\tt 2 3}


\newpage
\Q{12} Consider the following coin changing problem. We have a set of coin denominations $d_1 > d_2 > .... >d_n = 1$, such that $d_{j}$ is divisible by $d_{j+1}$ for all $1 \le j \le n-1$, and a target sum $T_0$. As in real life, we assume that $d_i$ are positive integers. Prove that the greedy ``cashier algorithm'' that gives back the maximum number of coins using the highest valued denominator next is optimal. You can think of any algorithm solving this problem as giving back a tuple $[x_1, ..., x_n]$, such that  $x_1d_1 + x_2d_2 + .... + x_nd_n=T_0$. The greedy cashier algorithm first gives $x_1 = \lfloor \frac{T_0}{d_1}\rfloor$ many $d_1$ coins. Then let $T_1 = T_0 - x_1d_1$. Then the algorithm gives $x_2=\lfloor \frac{T_1}{d_2}\rfloor$ many $d_2$ coins, so on and so forth.

For the rest of the question, let $S_g=[x_1, ..., x_n]$ be the greedy solution.  
\begin{enumerate}
\Q{2} Let $f_i = \frac{d_{i-1}}{d_i}$ for $i \ge 2$. Show that in $S_g$, $0 \le x_i < f_i$ for all $i \ge 2$.

Apperantly, $x_i \leq 0$ for all $i$. Assume there exists an $x_i$ such that $x_i \geq f_i$. So 
$x_i \geq \frac{d_{i - 1}}{d_i}$, $d_i x_i \geq d_{i - 1}$ since $d_i$ is positive. Since $x_i \leq 
\frac {T_i} {d_{i - 1}}$, $d_i x_i \leq T_{i - 1}$. Hence $d_{i - 1} \leq T_{i - 1}$. However, 
\begin{align*}
    T_{i - 1} & = T_{i - 2} - x_{i - 1} d_{i - 1}\\
    & =  T_{i - 2} - \lfloor \frac {T_{i - 2}} {d_{i - 1}} \rfloor d_{i - 1}\\
    & < T_{i - 2} - (\frac {T_{i - 2}} {d_{i - 1}} - 1)  d_{i - 1}\\
    & = d_{i - 1}
\end{align*}
 a contradiction. 

\Q{2} Let $S=[y_1, ..., y_n]$ be an arbitrary solution. Show that if $y_i \ge f_i$ for any $i \ge 2$, then $S$ is not optimal. Note that this proves that if $S$ is an optimal solution, similar to the property of $S_g$ we proved, $0 \le y_i < f_i$, for $i \ge 2$. 

Since $d_{i - 1}$ is dividsible by $d_i$, let $d_{i - 1} = k d_i$ for some integer $k > 1$. 
\begin{align*}
    y_{i - 1} d_{i - 1} + y_i d_i & = y_{i - 1} d_{i - 1} + d_{i - 1} + y_i d_i - d_{i - 1}\\
    & = ( y_{i - 1} + 1) d_{i - 1} + y_i d_i - k d_i\\
    & = ( y_{i - 1} + 1) d_{i - 1} + (y_i - k) d_i
\end{align*}
Since $d_i y_i \geq d_{i - 1}$, $(y_i - k) d_i$ is positive.
Therefore, 
\begin{align*}
    T_0 & = y_1 d_1 + \cdots + y_{i - 1} d_{i - 1} + y_i d_i + \cdots + y_n d_n\\
    & = y_1 d_1 + \cdots + ( y_{i - 1} + 1) d_{i - 1} + (y_i - k) d_i + \cdots + y_n d_n
\end{align*}
But $y_{i - 1} + y_i > y_{i - 1} + 1 - k$ since $k > 1$. Therefore $S$ is not optimal.


\Q{8} Now suppose that $S=[y_1, ..., y_n]$ is a hypothetical optimal solution, so $0 \le y_i < f_i$, for $i \ge 2$, but for the purpose of contradiction, $S \neq S_g$. Let $i^*$ be the {\em highest} index (so the smallest denomination) where $S_g$ and $S$ differ. Derive a contradiction that $S$ is optimal. This completes the proof that $S_g$ is the optimal solution.

Let $K = \{i: x_i \neq y_i\}$. Since $d_i$ is divisible by $d_{i + 1}$ and $d_{i^*}$ is smallest in $K$, 
$d_k = c_k d_i$ for all $k \in K$. \\
Also $\sum_{i = 1}^{n} y_i d_i  = \sum_{i = 1}^{n} x_i d_i$, $\sum_{k \in K} (x_k - y_k) d_k = 0$ \\
\begin{align*}
    (x_{i^*} - y_{i^*}) d_{i^*} + \sum_{k \in K \backslash \{i^*\}} (x_k - y_k) d_k & = 0 \\
    (x_{i^*} - y_{i^*}) d_{i^*} & = - d_{i^* - 1} \sum_{k \in K \backslash \{i^*\}} c_k(x_k - y_k)\\
    \lvert x_{i^*} - y_{i^*} \rvert & = f_{i^*} \sum_{k \in K \backslash \{i^*\}} \lvert c_k(x_k - y_k) \rvert\\
    \lvert x_{i^*} - y_{i^*} \rvert & > f_{i^*} \text{\quad since } c_k > 1, |x_k - y_k| > 1  
\end{align*}
However, $y_{i^*} < f_{i^*}$ and $x_{i^*} < f_{i^*}$, $\lvert x_{i^*} - y_{i^*} \rvert < f_{i^*}$, a contradiction.

\end{enumerate}


\newpage
\Q{13} Consider the input for the activity selection problem that we covered in Lecture 7. The input consists of a set of line segments $r_1, ..., r_n$, with start times $s(i)$ and finishing times $f(i)$. Now consider the version of the problem where instead of trying to pick the maximum number of non-overlapping line segments, we want to pick the minimum number of line segments $S$ such that each line segment $r_i$ overlaps with at least one of the line segments in $S$. Develop a greedy algorithm for this problem. Your algorithm can be $O(n^2)$ time or faster, e.g., $O(n\log n)$. You will not lose points if your algorithm is $O(n^2)$ (but you will lose points if it's slower than $O(n^2)$.)

Let $overlap(r_i)$ denot the set of intervals than overlaps with $r_i$(which includes $r_i$ itself).

\begin{algorithm} [h]
    \caption{overlapLs($R[r_1, \dots, r_n]$)}
    sort($R$) according to $f(i)$ in ascending order\\
    $S = \emptyset$\\
    \While {$\lvert R \rvert > 1$} {
        $A = overlap(r)$\\
        $S = S \cup {a}$ where $a$ has the latest ending time in $A$\\
        $R = R \backslash overlap(r)$
    }
    \uIf{$\lvert R \rvert = 1$ \textbf{and} $R[0]$ does not overlap with some segment in $S$} {
        \Return{$S \cup \{R[0]\}$}
    } \Else{
        \Return{$S$}
    }
\end{algorithm}

\textbf{Proof of Optimality and Correctness:} At each iteration, we only remove the line segments that 
overlap with $r$. We claim that $a$ overlaps with all line segments in $overlap(r)$. If there exists a line 
segment $r'$ such that $f(r') < s(a)$, then before this iteration, $r$ is already in $S$ not in $R$, a 
contradiction. Since $a$ overlaps with all line segments in $overlap(r)$, this is correct.\\
At each iteration, the greedy algorithm selects the line segment that overlaps with most segments containing the 
one with earilest starting time in $R$. So by construction, $a$ is the longest line segment that also contains 
the one with earilest finishing time. Hence we cannot find another segment that overlaps as many as $a$ can and 
also contains the one with earilest finishing time. Therefore, the greedy algorithm always stays ahead.

\textbf{Runtime Analysis:} Sorting takes $O(n \log n)$ time. The while loop iterates at most $n$ times, and takes 
$O(n)$ time to find all overlapping intervals of $a$ for each $r \in R$. 
So the total runtime is $T(n) \leq O(n \log n) + n O(n) = O(n^2)$.

\newpage
\Q{12} There is an election that will take place in your country between two parties, party $A$ and party $B$. The election takes place in $n$ states and you have calculated the probability of party $A$ winning in each state $i$ as $a_i$ (so party $B$ winning in state $i$ is ($1-a_i$)). You can assume that the outcomes of the elections in each state is independent from each other. Given the $n$ probabilities of party $A$ winning in each state, $a_1, ..., a_n$, design an $O(n^2)$ dynamic programming algorithm that computes the probability of party $A$ winning the general election, i.e., the probability that $A$ wins in strictly more than $n/2$ states.

Here are two facts from probability theory that will be needed for this problem: 

\noindent {\em Independence:} When two events $e_1$ and $e_2$ are independent, then the probability of both events happening is the multiplication of their probabilities. For example, in our example, the probability of party $A$ winning in states 1 and 2 are $a_1a_2$. 

\noindent {\em Union/OR of Mutually Exclusive Events:} The probability of one of two mutually exclusive events $e_1$ and $e_2$ is happening the sum of their probabilities. Two events are mutually exclusive if they cannot occur at the same time. For example, let $e_1$ be the probability of party $A$ winning in state 1 and state 2, which happens (by independence) with probability $a_1*a_2$. Let $e_2$ be the event that party $A$ loses in both state 1 and state 2, which happens with probability $(1-a_1)(1-a_2)$. The probability that party $A$ either wins or loses in states 1 and 2 is $a_1a_2$ + $(1-a_1)(1-a_2)$.

\begin{algorithm} [h]
    \caption{ProbAWin($a_1, \dots, a_n$)}
    $P = 0_{(n + 1) \times (n + 1)}$\\
    $P[0, 0] = 1$\\
    $P[0,i] = \Pi_{j = 1}^{i} (1 - a_j)$ for $i \neq 0$\\
    \For{$i = 1, \dots, n$}{
        \For{$j = 1, \dots, i$} {
            $P[i, j] = P[i - 1, j - 1] \times a_j + P[i - 1, j] \times (1 - a_j)$\\
        }
    }
    \Return{$sum(P[\lceil n / 2 \rceil, n], P[\lceil n / 2 \rceil + 1, n], \dots, p[n, n])$}
\end{algorithm} 

\textbf{Proof of Correctness:} Let $(A, i, j)$ be the event that $A$ wins in $j$ states with $i$ states in total. 
Since for each state $j$, the outcome is either win or loose. 
\begin{align*}
    P((A, i, j)) & = P((A, i - 1, j - 1)) \times 
                    P(\text{state j wins}) + P((A, i - 1, j)) \times P(\text{state j loses}) \\
    & = P((A, i - 1, j - 1)) \times a_j + P((A, i - 1, j)) \times (1 - a_j)
\end{align*}
And, $P(\text{A wins in strictly more than n / 2 states}) = P((A, \lceil n / 2 \rceil, n)) + 
P((A, \lceil n / 2 \rceil + 1, n)) + \cdots + P((A, n, n))$.

\textbf{Runtime Analysis:} $T(n) = O(n^2) + O(n) = O(n^2)$
\end{enumerate}

\end{document}

